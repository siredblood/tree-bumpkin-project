\section{\module{calendar} ---
         General calendar-related functions}

\declaremodule{standard}{calendar}
\modulesynopsis{Functions for working with calendars,
                including some emulation of the \UNIX\ \program{cal}
                program.}
\sectionauthor{Drew Csillag}{drew_csillag@geocities.com}

This module allows you to output calendars like the \UNIX{}
\program{cal} program, and provides additional useful functions
related to the calendar. By default, these calendars have Monday as
the first day of the week, and Sunday as the last (the European
convention). Use \function{setfirstweekday()} to set the first day of the
week to Sunday (6) or to any other weekday.  Parameters that specify
dates are given as integers.

Most of these functions and classses rely on the \module{datetime}
module which uses an idealized calendar, the current Gregorian
calendar indefinitely extended in both directions.  This matches
the definition of the "proleptic Gregorian" calendar in Dershowitz
and Reingold's book "Calendrical Calculations", where it's the
base calendar for all computations.

\begin{classdesc}{Calendar}{\optional{firstweekday}}
Creates a \class{Calendar} object. \var{firstweekday} is an integer
specifying the first day of the week. \code{0} is Monday (the default),
\code{6} is Sunday.

A \class{Calendar} object provides several methods that can
be used for preparing the calendar data for formatting. This
class doesn't do any formatting itself. This is the job of
subclasses.
\versionadded{2.5}
\end{classdesc}

\class{Calendar} instances have the following methods:

\begin{methoddesc}{iterweekdays}{weekday}
Return an iterator for the week day numbers that will be used
for one week. The first value from the iterator will be the same
as the value of the \member{firstweekday} property.
\end{methoddesc}

\begin{methoddesc}{itermonthdates}{year, month}
Return an iterator for the month \var{month} (1-12) in the
year \var{year}. This iterator will return all days (as
\class{datetime.date} objects) for the month and all days
before the start of the month or after the end of the month
that are required to get a complete week.
\end{methoddesc}

\begin{methoddesc}{itermonthdays2}{year, month}
Return an iterator for the month \var{month} in the year
\var{year} similar to \method{itermonthdates()}. Days returned
will be tuples consisting of a day number and a week day
number.
\end{methoddesc}

\begin{methoddesc}{itermonthdays}{year, month}
Return an iterator for the month \var{month} in the year
\var{year} similar to \method{itermonthdates()}. Days returned
will simply be day numbers.
\end{methoddesc}

\begin{methoddesc}{monthdatescalendar}{year, month}
Return a list of the weeks in the month \var{month} of
the \var{year} as full weeks. Weeks are lists of seven
\class{datetime.date} objects.
\end{methoddesc}

\begin{methoddesc}{monthdays2calendar}{year, month}
Return a list of the weeks in the month \var{month} of
the \var{year} as full weeks. Weeks are lists of seven
tuples of day numbers and weekday numbers.
\end{methoddesc}

\begin{methoddesc}{monthdayscalendar}{year, month}
Return a list of the weeks in the month \var{month} of
the \var{year} as full weeks. Weeks are lists of seven
day numbers.
\end{methoddesc}

\begin{methoddesc}{yeardatescalendar}{year\optional{, width}}
Return the data for the specified year ready for formatting. The return
value is a list of month rows. Each month row contains up to \var{width}
months (defaulting to 3). Each month contains between 4 and 6 weeks and
each week contains 1--7 days. Days are \class{datetime.date} objects.
\end{methoddesc}

\begin{methoddesc}{yeardays2calendar}{year\optional{, width}}
Return the data for the specified year ready for formatting (similar to
\method{yeardatescalendar()}). Entries in the week lists are tuples of
day numbers and weekday numbers. Day numbers outside this month are zero.
\end{methoddesc}

\begin{methoddesc}{yeardayscalendar}{year\optional{, width}}
Return the data for the specified year ready for formatting (similar to
\method{yeardatescalendar()}). Entries in the week lists are day numbers.
Day numbers outside this month are zero.
\end{methoddesc}


\begin{classdesc}{TextCalendar}{\optional{firstweekday}}
This class can be used to generate plain text calendars.

\versionadded{2.5}
\end{classdesc}

\class{TextCalendar} instances have the following methods:

\begin{methoddesc}{formatmonth}{theyear, themonth\optional{, w\optional{, l}}}
Return a month's calendar in a multi-line string. If \var{w} is
provided, it specifies the width of the date columns, which are
centered. If \var{l} is given, it specifies the number of lines that
each week will use. Depends on the first weekday as specified in the constructor
or set by the \method{setfirstweekday()} method.
\end{methoddesc}

\begin{methoddesc}{prmonth}{theyear, themonth\optional{, w\optional{, l}}}
Print a month's calendar as returned by \method{formatmonth()}.
\end{methoddesc}

\begin{methoddesc}{formatyear}{theyear, themonth\optional{, w\optional{,
                               l\optional{, c\optional{, m}}}}}
Return a \var{m}-column calendar for an entire year as a multi-line string.
Optional parameters \var{w}, \var{l}, and \var{c} are for date column
width, lines per week, and number of spaces between month columns,
respectively. Depends on the first weekday as specified in the constructor or
set by the \method{setfirstweekday()} method.  The earliest year for which a
calendar can be generated is platform-dependent.
\end{methoddesc}

\begin{methoddesc}{pryear}{theyear\optional{, w\optional{, l\optional{,
                           c\optional{, m}}}}}
Print the calendar for an entire year as returned by \method{formatyear()}.
\end{methoddesc}


\begin{classdesc}{HTMLCalendar}{\optional{firstweekday}}
This class can be used to generate HTML calendars.

\versionadded{2.5}
\end{classdesc}

\class{HTMLCalendar} instances have the following methods:

\begin{methoddesc}{formatmonth}{theyear, themonth\optional{, withyear}}
Return a month's calendar as an HTML table. If \var{withyear} is
true the year will be included in the header, otherwise just the
month name will be used.
\end{methoddesc}

\begin{methoddesc}{formatyear}{theyear, themonth\optional{, width}}
Return a year's calendar as an HTML table. \var{width} (defaulting to 3)
specifies the number of months per row.
\end{methoddesc}

\begin{methoddesc}{formatyearpage}{theyear, themonth\optional{,
                                   width\optional{, css\optional{, encoding}}}}
Return a year's calendar as a complete HTML page. \var{width}
(defaulting to 3) specifies the number of months per row. \var{css}
is the name for the cascading style sheet to be used. \constant{None}
can be passed if no style sheet should be used. \var{encoding}
specifies the encoding to be used for the output (defaulting
to the system default encoding).
\end{methoddesc}


\begin{classdesc}{LocaleTextCalendar}{\optional{firstweekday\optional{, locale}}}
This subclass of \class{TextCalendar} can be passed a locale name in the
constructor and will return month and weekday names in the specified locale.
If this locale includes an encoding all strings containing month and weekday
names will be returned as unicode.
\versionadded{2.5}
\end{classdesc}


\begin{classdesc}{LocaleHTMLCalendar}{\optional{firstweekday\optional{, locale}}}
This subclass of \class{HTMLCalendar} can be passed a locale name in the
constructor and will return month and weekday names in the specified locale.
If this locale includes an encoding all strings containing month and weekday
names will be returned as unicode.
\versionadded{2.5}
\end{classdesc}


For simple text calendars this module provides the following functions.

\begin{funcdesc}{setfirstweekday}{weekday}
Sets the weekday (\code{0} is Monday, \code{6} is Sunday) to start
each week. The values \constant{MONDAY}, \constant{TUESDAY},
\constant{WEDNESDAY}, \constant{THURSDAY}, \constant{FRIDAY},
\constant{SATURDAY}, and \constant{SUNDAY} are provided for
convenience. For example, to set the first weekday to Sunday:

\begin{verbatim}
import calendar
calendar.setfirstweekday(calendar.SUNDAY)
\end{verbatim}
\versionadded{2.0}
\end{funcdesc}

\begin{funcdesc}{firstweekday}{}
Returns the current setting for the weekday to start each week.
\versionadded{2.0}
\end{funcdesc}

\begin{funcdesc}{isleap}{year}
Returns \constant{True} if \var{year} is a leap year, otherwise
\constant{False}.
\end{funcdesc}

\begin{funcdesc}{leapdays}{y1, y2}
Returns the number of leap years in the range
[\var{y1}\ldots\var{y2}), where \var{y1} and \var{y2} are years.
\versionchanged[This function didn't work for ranges spanning 
                a century change in Python 1.5.2]{2.0}
\end{funcdesc}

\begin{funcdesc}{weekday}{year, month, day}
Returns the day of the week (\code{0} is Monday) for \var{year}
(\code{1970}--\ldots), \var{month} (\code{1}--\code{12}), \var{day}
(\code{1}--\code{31}).
\end{funcdesc}

\begin{funcdesc}{weekheader}{n}
Return a header containing abbreviated weekday names. \var{n} specifies
the width in characters for one weekday.
\end{funcdesc}

\begin{funcdesc}{monthrange}{year, month}
Returns weekday of first day of the month and number of days in month, 
for the specified \var{year} and \var{month}.
\end{funcdesc}

\begin{funcdesc}{monthcalendar}{year, month}
Returns a matrix representing a month's calendar.  Each row represents
a week; days outside of the month a represented by zeros.
Each week begins with Monday unless set by \function{setfirstweekday()}.
\end{funcdesc}

\begin{funcdesc}{prmonth}{theyear, themonth\optional{, w\optional{, l}}}
Prints a month's calendar as returned by \function{month()}.
\end{funcdesc}

\begin{funcdesc}{month}{theyear, themonth\optional{, w\optional{, l}}}
Returns a month's calendar in a multi-line string using the
\method{formatmonth} of the \class{TextCalendar} class.
\versionadded{2.0}
\end{funcdesc}

\begin{funcdesc}{prcal}{year\optional{, w\optional{, l\optional{c}}}}
Prints the calendar for an entire year as returned by 
\function{calendar()}.
\end{funcdesc}

\begin{funcdesc}{calendar}{year\optional{, w\optional{, l\optional{c}}}}
Returns a 3-column calendar for an entire year as a multi-line string
using the \method{formatyear} of the \class{TextCalendar} class.
\versionadded{2.0}
\end{funcdesc}

\begin{funcdesc}{timegm}{tuple}
An unrelated but handy function that takes a time tuple such as
returned by the \function{gmtime()} function in the \refmodule{time}
module, and returns the corresponding \UNIX{} timestamp value, assuming
an epoch of 1970, and the POSIX encoding.  In fact,
\function{time.gmtime()} and \function{timegm()} are each others' inverse.
\versionadded{2.0}
\end{funcdesc}

The \module{calendar} module exports the following data attributes:

\begin{datadesc}{day_name}
An array that represents the days of the week in the
current locale.
\end{datadesc}

\begin{datadesc}{day_abbr}
An array that represents the abbreviated days of the week
in the current locale.
\end{datadesc}

\begin{datadesc}{month_name}
An array that represents the months of the year in the
current locale.  This follows normal convention
of January being month number 1, so it has a length of 13 and 
\code{month_name[0]} is the empty string.
\end{datadesc}

\begin{datadesc}{month_abbr}
An array that represents the abbreviated months of the year
in the current locale.  This follows normal convention
of January being month number 1, so it has a length of 13 and 
\code{month_abbr[0]} is the empty string.
\end{datadesc}

\begin{seealso}
  \seemodule{datetime}{Object-oriented interface to dates and times
                       with similar functionality to the
                       \refmodule{time} module.}
  \seemodule{time}{Low-level time related functions.}
\end{seealso}
